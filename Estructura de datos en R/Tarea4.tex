% Options for packages loaded elsewhere
\PassOptionsToPackage{unicode}{hyperref}
\PassOptionsToPackage{hyphens}{url}
%
\documentclass[
]{article}
\usepackage{lmodern}
\usepackage{amssymb,amsmath}
\usepackage{ifxetex,ifluatex}
\ifnum 0\ifxetex 1\fi\ifluatex 1\fi=0 % if pdftex
  \usepackage[T1]{fontenc}
  \usepackage[utf8]{inputenc}
  \usepackage{textcomp} % provide euro and other symbols
\else % if luatex or xetex
  \usepackage{unicode-math}
  \defaultfontfeatures{Scale=MatchLowercase}
  \defaultfontfeatures[\rmfamily]{Ligatures=TeX,Scale=1}
\fi
% Use upquote if available, for straight quotes in verbatim environments
\IfFileExists{upquote.sty}{\usepackage{upquote}}{}
\IfFileExists{microtype.sty}{% use microtype if available
  \usepackage[]{microtype}
  \UseMicrotypeSet[protrusion]{basicmath} % disable protrusion for tt fonts
}{}
\makeatletter
\@ifundefined{KOMAClassName}{% if non-KOMA class
  \IfFileExists{parskip.sty}{%
    \usepackage{parskip}
  }{% else
    \setlength{\parindent}{0pt}
    \setlength{\parskip}{6pt plus 2pt minus 1pt}}
}{% if KOMA class
  \KOMAoptions{parskip=half}}
\makeatother
\usepackage{xcolor}
\IfFileExists{xurl.sty}{\usepackage{xurl}}{} % add URL line breaks if available
\IfFileExists{bookmark.sty}{\usepackage{bookmark}}{\usepackage{hyperref}}
\hypersetup{
  pdftitle={Ejercicios sobre LaTeX, R y Markdown},
  pdfauthor={Alexis Frías Domínguez},
  hidelinks,
  pdfcreator={LaTeX via pandoc}}
\urlstyle{same} % disable monospaced font for URLs
\usepackage[margin=1in]{geometry}
\usepackage{color}
\usepackage{fancyvrb}
\newcommand{\VerbBar}{|}
\newcommand{\VERB}{\Verb[commandchars=\\\{\}]}
\DefineVerbatimEnvironment{Highlighting}{Verbatim}{commandchars=\\\{\}}
% Add ',fontsize=\small' for more characters per line
\usepackage{framed}
\definecolor{shadecolor}{RGB}{248,248,248}
\newenvironment{Shaded}{\begin{snugshade}}{\end{snugshade}}
\newcommand{\AlertTok}[1]{\textcolor[rgb]{0.94,0.16,0.16}{#1}}
\newcommand{\AnnotationTok}[1]{\textcolor[rgb]{0.56,0.35,0.01}{\textbf{\textit{#1}}}}
\newcommand{\AttributeTok}[1]{\textcolor[rgb]{0.77,0.63,0.00}{#1}}
\newcommand{\BaseNTok}[1]{\textcolor[rgb]{0.00,0.00,0.81}{#1}}
\newcommand{\BuiltInTok}[1]{#1}
\newcommand{\CharTok}[1]{\textcolor[rgb]{0.31,0.60,0.02}{#1}}
\newcommand{\CommentTok}[1]{\textcolor[rgb]{0.56,0.35,0.01}{\textit{#1}}}
\newcommand{\CommentVarTok}[1]{\textcolor[rgb]{0.56,0.35,0.01}{\textbf{\textit{#1}}}}
\newcommand{\ConstantTok}[1]{\textcolor[rgb]{0.00,0.00,0.00}{#1}}
\newcommand{\ControlFlowTok}[1]{\textcolor[rgb]{0.13,0.29,0.53}{\textbf{#1}}}
\newcommand{\DataTypeTok}[1]{\textcolor[rgb]{0.13,0.29,0.53}{#1}}
\newcommand{\DecValTok}[1]{\textcolor[rgb]{0.00,0.00,0.81}{#1}}
\newcommand{\DocumentationTok}[1]{\textcolor[rgb]{0.56,0.35,0.01}{\textbf{\textit{#1}}}}
\newcommand{\ErrorTok}[1]{\textcolor[rgb]{0.64,0.00,0.00}{\textbf{#1}}}
\newcommand{\ExtensionTok}[1]{#1}
\newcommand{\FloatTok}[1]{\textcolor[rgb]{0.00,0.00,0.81}{#1}}
\newcommand{\FunctionTok}[1]{\textcolor[rgb]{0.00,0.00,0.00}{#1}}
\newcommand{\ImportTok}[1]{#1}
\newcommand{\InformationTok}[1]{\textcolor[rgb]{0.56,0.35,0.01}{\textbf{\textit{#1}}}}
\newcommand{\KeywordTok}[1]{\textcolor[rgb]{0.13,0.29,0.53}{\textbf{#1}}}
\newcommand{\NormalTok}[1]{#1}
\newcommand{\OperatorTok}[1]{\textcolor[rgb]{0.81,0.36,0.00}{\textbf{#1}}}
\newcommand{\OtherTok}[1]{\textcolor[rgb]{0.56,0.35,0.01}{#1}}
\newcommand{\PreprocessorTok}[1]{\textcolor[rgb]{0.56,0.35,0.01}{\textit{#1}}}
\newcommand{\RegionMarkerTok}[1]{#1}
\newcommand{\SpecialCharTok}[1]{\textcolor[rgb]{0.00,0.00,0.00}{#1}}
\newcommand{\SpecialStringTok}[1]{\textcolor[rgb]{0.31,0.60,0.02}{#1}}
\newcommand{\StringTok}[1]{\textcolor[rgb]{0.31,0.60,0.02}{#1}}
\newcommand{\VariableTok}[1]{\textcolor[rgb]{0.00,0.00,0.00}{#1}}
\newcommand{\VerbatimStringTok}[1]{\textcolor[rgb]{0.31,0.60,0.02}{#1}}
\newcommand{\WarningTok}[1]{\textcolor[rgb]{0.56,0.35,0.01}{\textbf{\textit{#1}}}}
\usepackage{graphicx,grffile}
\makeatletter
\def\maxwidth{\ifdim\Gin@nat@width>\linewidth\linewidth\else\Gin@nat@width\fi}
\def\maxheight{\ifdim\Gin@nat@height>\textheight\textheight\else\Gin@nat@height\fi}
\makeatother
% Scale images if necessary, so that they will not overflow the page
% margins by default, and it is still possible to overwrite the defaults
% using explicit options in \includegraphics[width, height, ...]{}
\setkeys{Gin}{width=\maxwidth,height=\maxheight,keepaspectratio}
% Set default figure placement to htbp
\makeatletter
\def\fps@figure{htbp}
\makeatother
\setlength{\emergencystretch}{3em} % prevent overfull lines
\providecommand{\tightlist}{%
  \setlength{\itemsep}{0pt}\setlength{\parskip}{0pt}}
\setcounter{secnumdepth}{-\maxdimen} % remove section numbering

\title{Ejercicios sobre LaTeX, R y Markdown}
\author{Alexis Frías Domínguez}
\date{10/8/2020}

\begin{document}
\maketitle

\hypertarget{instrucciones}{%
\section{Instrucciones}\label{instrucciones}}

En primer lugar, debéis reproducir este documento tal cual está.
Necesitaréis instalar MiKTeX y Texmaker. A continuación de cada
pregunta, tenéis que redactar vuestras respuestas de manera correcta y
argumentada, indicando qué hacéis, por qué, etc. Si se os pide utilizar
instrucciones de \texttt{R}, tendréis que mostrarlas todas en chunks. El
objetivo de esta tarea es que os familiaricéis con los documentos
Markdown, las fórmulas en \LaTeX y los chunks de \texttt{R}. Y, de lo
más importante, que os acostumbréis a explicar lo que hacéis en cada
momento.

\hypertarget{preguntas}{%
\section{Preguntas}\label{preguntas}}

\hypertarget{pregunta-1}{%
\subsection{Pregunta 1}\label{pregunta-1}}

Realizad los siguientes productos de matrices siguiente en \texttt{R}:

\begin{center}
$A\cdot B$

$B\cdot A$

$(A\cdot B)^t$

$B^t\cdot A$

$(A\cdot B)^{-1}$

$A^{-1}\cdot B^t$
\end{center}

Donde:

\begin{equation}
A =
\begin{pmatrix}
1 & 2 & 3 & 4\\
4 & 3 & 2 & 1\\
0 & 1 & 0 & 2\\
3 & 0 & 4 & 0
\end{pmatrix}
B = 
\begin{pmatrix}
4 & 3 & 2 & 1\\
0 & 3 & 0 & 4\\
1 & 2 & 3 & 4\\
0 & 1 & 0 & 2 
\end{pmatrix}
\end{equation}

Definimos las matrices A y B:

\begin{Shaded}
\begin{Highlighting}[]
\NormalTok{A =}\StringTok{ }\KeywordTok{matrix}\NormalTok{(}\KeywordTok{c}\NormalTok{(}\DecValTok{1}\NormalTok{,}\DecValTok{2}\NormalTok{,}\DecValTok{3}\NormalTok{,}\DecValTok{4}\NormalTok{,}\DecValTok{4}\NormalTok{,}\DecValTok{3}\NormalTok{,}\DecValTok{2}\NormalTok{,}\DecValTok{1}\NormalTok{,}\DecValTok{0}\NormalTok{,}\DecValTok{1}\NormalTok{,}\DecValTok{0}\NormalTok{,}\DecValTok{2}\NormalTok{,}\DecValTok{3}\NormalTok{,}\DecValTok{0}\NormalTok{,}\DecValTok{4}\NormalTok{,}\DecValTok{0}\NormalTok{),}\DataTypeTok{ncol =} \DecValTok{4}\NormalTok{, }\DataTypeTok{byrow =}\NormalTok{ T)}
\NormalTok{A}
\end{Highlighting}
\end{Shaded}

\begin{verbatim}
##      [,1] [,2] [,3] [,4]
## [1,]    1    2    3    4
## [2,]    4    3    2    1
## [3,]    0    1    0    2
## [4,]    3    0    4    0
\end{verbatim}

\begin{Shaded}
\begin{Highlighting}[]
\NormalTok{B =}\StringTok{ }\KeywordTok{matrix}\NormalTok{(}\KeywordTok{c}\NormalTok{(}\DecValTok{4}\NormalTok{,}\DecValTok{3}\NormalTok{,}\DecValTok{2}\NormalTok{,}\DecValTok{1}\NormalTok{,}\DecValTok{0}\NormalTok{,}\DecValTok{3}\NormalTok{,}\DecValTok{0}\NormalTok{,}\DecValTok{4}\NormalTok{,}\DecValTok{1}\NormalTok{,}\DecValTok{2}\NormalTok{,}\DecValTok{3}\NormalTok{,}\DecValTok{4}\NormalTok{,}\DecValTok{0}\NormalTok{,}\DecValTok{1}\NormalTok{,}\DecValTok{0}\NormalTok{,}\DecValTok{2}\NormalTok{),}\DataTypeTok{ncol =} \DecValTok{4}\NormalTok{, }\DataTypeTok{byrow =}\NormalTok{ T)}
\NormalTok{B}
\end{Highlighting}
\end{Shaded}

\begin{verbatim}
##      [,1] [,2] [,3] [,4]
## [1,]    4    3    2    1
## [2,]    0    3    0    4
## [3,]    1    2    3    4
## [4,]    0    1    0    2
\end{verbatim}

Realizamos \(A\cdot B\)

\begin{Shaded}
\begin{Highlighting}[]
\NormalTok{A}\OperatorTok{*}\NormalTok{B}
\end{Highlighting}
\end{Shaded}

\begin{verbatim}
##      [,1] [,2] [,3] [,4]
## [1,]    4    6    6    4
## [2,]    0    9    0    4
## [3,]    0    2    0    8
## [4,]    0    0    0    0
\end{verbatim}

Realizamos \(B\cdot A\)

\begin{Shaded}
\begin{Highlighting}[]
\NormalTok{B}\OperatorTok{*}\NormalTok{A}
\end{Highlighting}
\end{Shaded}

\begin{verbatim}
##      [,1] [,2] [,3] [,4]
## [1,]    4    6    6    4
## [2,]    0    9    0    4
## [3,]    0    2    0    8
## [4,]    0    0    0    0
\end{verbatim}

Realizamos \((A\cdot B)^t\)

\begin{Shaded}
\begin{Highlighting}[]
\KeywordTok{t}\NormalTok{((A}\OperatorTok{*}\NormalTok{B))}
\end{Highlighting}
\end{Shaded}

\begin{verbatim}
##      [,1] [,2] [,3] [,4]
## [1,]    4    0    0    0
## [2,]    6    9    2    0
## [3,]    6    0    0    0
## [4,]    4    4    8    0
\end{verbatim}

\(B^t\cdot A\)

\begin{Shaded}
\begin{Highlighting}[]
\KeywordTok{t}\NormalTok{(B)}\OperatorTok{*}\NormalTok{A}
\end{Highlighting}
\end{Shaded}

\begin{verbatim}
##      [,1] [,2] [,3] [,4]
## [1,]    4    0    3    0
## [2,]   12    9    4    1
## [3,]    0    0    0    0
## [4,]    3    0   16    0
\end{verbatim}

Realizamos \((A\cdot B)^{-1}\)

\begin{Shaded}
\begin{Highlighting}[]
\NormalTok{(A}\OperatorTok{*}\NormalTok{B)}\OperatorTok{^}\NormalTok{\{}\OperatorTok{-}\DecValTok{1}\NormalTok{\}}
\end{Highlighting}
\end{Shaded}

\begin{verbatim}
##      [,1]      [,2]      [,3]  [,4]
## [1,] 0.25 0.1666667 0.1666667 0.250
## [2,]  Inf 0.1111111       Inf 0.250
## [3,]  Inf 0.5000000       Inf 0.125
## [4,]  Inf       Inf       Inf   Inf
\end{verbatim}

Realizamos \(A^{-1}\cdot B^t\)

\begin{Shaded}
\begin{Highlighting}[]
\NormalTok{A}\OperatorTok{^}\NormalTok{\{}\OperatorTok{-}\DecValTok{1}\NormalTok{\}}\OperatorTok{*}\KeywordTok{t}\NormalTok{(B)}
\end{Highlighting}
\end{Shaded}

\begin{verbatim}
##           [,1] [,2]      [,3] [,4]
## [1,] 4.0000000    0 0.3333333    0
## [2,] 0.7500000    1 1.0000000    1
## [3,]       Inf    0       Inf    0
## [4,] 0.3333333  Inf 1.0000000  Inf
\end{verbatim}

\hypertarget{pregunta-2}{%
\subsection{Pregunta 2}\label{pregunta-2}}

Considerad en un vector los números de vuestro DNI y llamadlo dni. Por
ejemplo, si vuestro DNI es 54201567K, vuestro vector será

\begin{center}
$dni = (5, 4, 2, 0, 1, 5, 6, 7)$
\end{center}

Definid el vector en \texttt{R}. Calculad con \texttt{R} el vector dni
al cuadrado, la raíz cuadrada del vector dni y, por último, la suma de
todas las cifras del vector dni.

Finalmente, escribid todos estos vectores también a \LaTeX

Definiendo el vector

\begin{Shaded}
\begin{Highlighting}[]
\NormalTok{dni =}\StringTok{ }\KeywordTok{c}\NormalTok{(}\DecValTok{5}\NormalTok{,}\DecValTok{4}\NormalTok{,}\DecValTok{2}\NormalTok{,}\DecValTok{0}\NormalTok{,}\DecValTok{1}\NormalTok{,}\DecValTok{5}\NormalTok{,}\DecValTok{6}\NormalTok{,}\DecValTok{7}\NormalTok{)}
\end{Highlighting}
\end{Shaded}

Elevando al cuadrado

\begin{Shaded}
\begin{Highlighting}[]
\NormalTok{dni}\OperatorTok{^}\DecValTok{2}
\end{Highlighting}
\end{Shaded}

\begin{verbatim}
## [1] 25 16  4  0  1 25 36 49
\end{verbatim}

Calculando la raiz cuadrada

\begin{Shaded}
\begin{Highlighting}[]
\KeywordTok{sqrt}\NormalTok{(dni)}
\end{Highlighting}
\end{Shaded}

\begin{verbatim}
## [1] 2.236068 2.000000 1.414214 0.000000 1.000000 2.236068 2.449490 2.645751
\end{verbatim}

Sumando las entradas del vector

\begin{Shaded}
\begin{Highlighting}[]
\KeywordTok{sum}\NormalTok{(dni)}
\end{Highlighting}
\end{Shaded}

\begin{verbatim}
## [1] 30
\end{verbatim}

\hypertarget{pregunta-3}{%
\subsection{Pregunta 3}\label{pregunta-3}}

Considerad el vector de las letras de vuestro nombre y apellido.
Llamadlo name. Por ejemplo, en mi caso sería nombre = (M, A,R, I, A, S,
A,N, T,O, S). Definid dicho vector en \texttt{R}. Calculad el subvector
que solo contenga vuestro nombre. Calculad también el subvector que
contenga solo vuestro apellido. Ordenadlo alfabéticamente. Cread una
matriz con este vector.

Definiendo el vector

\begin{Shaded}
\begin{Highlighting}[]
\NormalTok{name =}\StringTok{ }\KeywordTok{c}\NormalTok{(}\StringTok{"A"}\NormalTok{,}\StringTok{"L"}\NormalTok{,}\StringTok{"E"}\NormalTok{,}\StringTok{"X"}\NormalTok{,}\StringTok{"I"}\NormalTok{,}\StringTok{"S"}\NormalTok{,}\StringTok{"F"}\NormalTok{,}\StringTok{"R"}\NormalTok{,}\StringTok{"I"}\NormalTok{,}\StringTok{"A"}\NormalTok{,}\StringTok{"S"}\NormalTok{)}
\NormalTok{name}
\end{Highlighting}
\end{Shaded}

\begin{verbatim}
##  [1] "A" "L" "E" "X" "I" "S" "F" "R" "I" "A" "S"
\end{verbatim}

Subvector con mi nombre

\begin{Shaded}
\begin{Highlighting}[]
\NormalTok{name[}\DecValTok{1}\OperatorTok{:}\DecValTok{6}\NormalTok{]}
\end{Highlighting}
\end{Shaded}

\begin{verbatim}
## [1] "A" "L" "E" "X" "I" "S"
\end{verbatim}

Subvector con mi apellido

\begin{Shaded}
\begin{Highlighting}[]
\NormalTok{name[}\DecValTok{7}\OperatorTok{:}\KeywordTok{length}\NormalTok{(name)]}
\end{Highlighting}
\end{Shaded}

\begin{verbatim}
## [1] "F" "R" "I" "A" "S"
\end{verbatim}

Ordenando el vector

\begin{Shaded}
\begin{Highlighting}[]
\KeywordTok{sort}\NormalTok{(name)}
\end{Highlighting}
\end{Shaded}

\begin{verbatim}
##  [1] "A" "A" "E" "F" "I" "I" "L" "R" "S" "S" "X"
\end{verbatim}

Creando la matriz

\begin{Shaded}
\begin{Highlighting}[]
\KeywordTok{matrix}\NormalTok{(name, }\DataTypeTok{ncol =} \DecValTok{6}\NormalTok{, }\DataTypeTok{nrow =}\NormalTok{ T)}
\end{Highlighting}
\end{Shaded}

\begin{verbatim}
## Warning in matrix(name, ncol = 6, nrow = T): la longitud de los datos [11] no es
## un submúltiplo o múltiplo del número de columnas [6] en la matriz
\end{verbatim}

\begin{verbatim}
##      [,1] [,2] [,3] [,4] [,5] [,6]
## [1,] "A"  "L"  "E"  "X"  "I"  "S"
\end{verbatim}

\end{document}
